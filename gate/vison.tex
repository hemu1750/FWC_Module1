\documentclass{article}
\usepackage{geometry}
\usepackage{enumitem}
\usepackege{
\usepackage{amsmath}
\usepackage{amssymb}
\begin{document}

\noindent
\textbf{Q.18} \quad Let \( f \) be a real-valued function of a real variable defined as 
\[
f(x) = x - \lfloor x \rfloor,
\]
where \( \lfloor x \rfloor \) denotes the largest integer less than or equal to \( x \). The value of 
\[
\int_{0.25}^{1.25} f(x) \, dx
\]
is

Given:
$ f(x) = x - [x], \quad [x] \ \text{is the greatest integer} \ \le x
We need:
\int_{0.25}^{1.25} f(x) \, dx
Step 1 — Understanding f(x)
 • x - [x] is the fractional part of x, denoted \{x\}.
 • \{x\} \in [0,1).
 • It repeats in each interval [n, n+1).
Step 2 — Splitting the Interval

The interval [0.25, 1.25] crosses the integer 1, so split at x = 1:
\int_{0.25}^{1.25} f(x) \, dx
= \int_{0.25}^{1} f(x) \, dx + \int_{1}^{1.25} f(x) \, dx

Step 3 — First Interval [0.25, 1)

For 0.25 \le x < 1:
[x] = 0 \quad \Rightarrow \quad f(x) = x
So:
\int_{0.25}^{1} x \, dx = \left[ \frac{x^2}{2} \right]_{0.25}^{1}
= \frac{1^2}{2} - \frac{(0.25)^2}{2}
= 0.5 - 0.03125
= 0.46875

Step 4 — Second Interval [1, 1.25]

For 1 \le x \le 1.25:
[x] = 1 \quad \Rightarrow \quad f(x) = x - 1
So:
\int_{1}^{1.25} (x - 1) \, dx
= \left[ \frac{x^2}{2} - x \right]_{1}^{1.25}
= \left(0.78125 - 1.25\right) - \left(0.5 - 1\right)
= -0.46875 - (-0.5)
= 0.03125

Step 5 — Adding Both Parts

\text{Total} = 0.46875 + 0.03125 = 0.50$
\end{document}
