\documentclass{article}
\usepackage{fontenc}
\usepackage{graphicx}
\usepackage[a4paper,margin=1in]{geometry}
\usepackage{amsmath,amssymb}
\usepackage{enumitem}
\usepackage{fancyhdr}
\usepackage{circuitikz}
\usepackage{titlesec}
\date{\today}
\geometry{top=1in, bottom=1in, left=1in, right=1in}
\pagestyle{empty}
\begin{document}
\thispagestyle{fancy}
\fancyhf{}
\fancyhead[L]{\includegraphics[width=8cm, height=1.7cm]{logo.png}}
\fancyhead[R]{

   Name:Hemanth Thatavarthi\\
   Batch:COMETFWC033\\
   Date:06 August 2025\\
}
\renewcommand{\headrulewidth}{0.4pt}
\fancyfoot[c]{\thepage}
\vspace{1cm}
\begin{center}
$$\textbf{\Huge GATE EE 2018  PAPER}$$
\end{center}
\section*{Question}

Let \(f\) be a real-valued function defined by
\[
f(x)=x-\lfloor x\rfloor,
\]
where \(\lfloor x\rfloor\) denotes the greatest integer less than or equal to \(x\). Compute
\[
\int_{0.25}^{1.25} f(x)\,dx
\]
(give the answer up to two decimal places).

\section*{Solution}

The function \(f(x)=x-\lfloor x\rfloor\) is the fractional part of \(x\). Split the integration interval at the integer \(x=1\):
\[
\int_{0.25}^{1.25} f(x)\,dx
=\int_{0.25}^{1} f(x)\,dx + \int_{1}^{1.25} f(x)\,dx.
\]

For \(0.25\le x<1\), \(\lfloor x\rfloor=0\) so \(f(x)=x\).  
For \(1\le x\le 1.25\), \(\lfloor x\rfloor=1\) so \(f(x)=x-1\).

Thus
\begin{align*}
\int_{0.25}^{1} f(x)\,dx &= \int_{0.25}^{1} x\,dx
= \left.\frac{x^2}{2}\right|_{0.25}^{1}
= \frac{1^2}{2}-\frac{0.25^2}{2}
= \frac{1}{2}-\frac{0.0625}{2}
= 0.5-0.03125 = 0.46875, \\[6pt]
\int_{1}^{1.25} f(x)\,dx &= \int_{1}^{1.25} (x-1)\,dx
= \left(\left.\frac{x^2}{2}\right|_{1}^{1.25}\right) - (1.25-1) \\
&= \left(\frac{1.5625}{2}-\frac{1}{2}\right)-0.25
= (0.78125-0.5)-0.25
= 0.28125-0.25 = 0.03125.
\end{align*}

Adding both parts:
\[
\int_{0.25}^{1.25} f(x)\,dx = 0.46875 + 0.03125 = 0.5.
\]

Therefore, up to two decimal places, the value is \(\boxed{0.50}\).
\end{document}
